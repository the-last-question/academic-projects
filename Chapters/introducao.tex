\chapter{Introdução}
\label{chap:intro}
\textbf{Vitor Silva: How can the net amount of entropy of the universe be massively decreased??}
Este pode ser um parágrafo citado por alguém \cite{Barabasi2003-1}.

%--------- NEW SECTION ----------------------
\section{Objetivos}
\label{sec:obj}

Nesta se\c{c}\~ao os objetivos principal (tamb\'em
pode-se se utilizar a palavra meta) da monografia de
gradua\c{c}\~ao ou especializa\c{c}\~ao, disserta\c{c}\~ao de
mestrado ou tese de doutorado s\~ao apresentados.


\subsection{Objetivos Específicos}
\label{ssec:objesp}

Nesta se\c{c}\~ao os objetivos espec\'ificos (tamb\'em
pode-se se utilizar a palavra meta) da monografia de
gradua\c{c}\~ao ou especializa\c{c}\~ao, disserta\c{c}\~ao de
mestrado ou tese de doutorado s\~ao apresentados.


%--------- NEW SECTION ----------------------
\section{Justificativa}
\label{sec:justi}

O pesquisador/estudante deve apresentar os aspectos mais
relevantes da pesquisa ressaltando os impactos (e.g. cient\'ifico,
tecnol\'ogico, econ\^omico, social e ambiental) que a pesquisa
causar\'a. Deve-se ter cuidado com a ingenuidade no momento em que
os argumentos forem apresentados.




%--------- NEW SECTION ----------------------
\section{Organização do documento}
\label{section:organizacao}

Este documento apresenta $5$ capítulos e está estruturado da seguinte forma:

\begin{itemize}

  \item \textbf{Capítulo \ref{chap:intro} - Introdução}: Contextualiza o âmbito, no qual a pesquisa proposta está inserida. Apresenta, portanto, a definição do problema, objetivos e justificativas da pesquisa e como este \thetypeworkthree está estruturado;
  \item \textbf{Capítulo \ref{chap:fundteor} - Fundamentação Teórica}: XXX;
  \item \textbf{Capítulo \ref{chap:mat} - Materiais e Métodos}: XXX;
  \item \textbf{Capítulo \ref{chap:result} - Resultados}: XXX;
  \item \textbf{Capítulo \ref{chap:conc} - Conclusão}: Apresenta as conclusóes, contribuições e algumas sugestões de atividades de pesquisa a serem desenvolvidas no futuro.

\end{itemize}
